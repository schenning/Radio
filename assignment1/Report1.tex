\documentclass{article}
\usepackage[utf8]{inputenc}
\usepackage{amsmath}
\title{Report Lab 1: Radio Engineering}
\author{Henning Schei}
\date{March 2016}
\usepackage{natbib}
\usepackage{graphicx}
\usepackage{listings}
\usepackage{color}
\usepackage{float}
\definecolor{dkgreen}{rgb}{0,0.6,0}
\definecolor{gray}{rgb}{0.5,0.5,0.5}
\definecolor{mauve}{rgb}{0.58,0,0.82}

\lstset{frame=tb,
  language=Matlab,
  aboveskip=3mm,
  belowskip=3mm,
  showstringspaces=false,
  columns=flexible,
  basicstyle={\small\ttfamily},
  numbers=none,
  numberstyle=\tiny\color{gray},
  keywordstyle=\color{blue},
  commentstyle=\color{dkgreen},
  stringstyle=\color{mauve},
  breaklines=true,
  breakatwhitespace=true,
  tabsize=3
}
\begin{document}
\maketitle

\section{Task 1}
\section{Task 2}
The plot shows the difference between the horizontal and vertical polarization. The reason why the horizontal wave are stronger may be the fact that the vertical wave are beeing attenuated by the surface. Since sea is not a smooth surface, it may be scattered and weaker. 

\section {Task 3}
		The model does fit the data very well. The charactheristic dip in the curve are als very accurate. .s










\end{document}



